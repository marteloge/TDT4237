\chapter{Configuration Management Testing}
	Often analysis of the infrastructure and topology architecture can reveal a great deal about a 
	web application. Information such as source code, HTTP methods permitted, administrative 
	functionality, authentication methods and infrastructural configurations can be obtained.

\clearpage
\section{SSL/TLS Testing}
	The http clear-text protocol is normally secured via an SSL or TLS tunnel, resulting in 
	https traffic. In addition to providing encryption of data in transit, https allows the 
	identification of servers (and, optionally, of clients) by means of digital certificates.

	In order to detect possible support of weak ciphers, the ports associated to SSL/TLS wrapped 
	services must be identified. These typically include port 443 which is the standard https port, 
	however this may change because a) https services may be configured to run on non-standard ports, 
	and b) there may be additional SSL/TLS wrapped services related to the web 
	application. In general a service discovery is required to identify such ports. Vulnerability 
	Scanners, in addition to performing service discovery, may include checks against weak ciphers.

	\begin{figure}[H]
		\centering
		\includegraphics{pics/ssl.jpg}
	\end{figure}

\section{DB Listener Testing}
	The Data base listener is a network daemon. It waits for connection requests from remote
	clients. This daemon can be compromised and hence can affect the availability of the database.

	It listens for connection requests and handles them accordingly. The listener, by default, 
	listens on port 1521(different ports may be used depending on the configuration). It is
	good practice to change the listener from this port to another arbitrary port number. 
	If this listener is "turned off" remote access to the database is not possible. If this 
	is the case, one’s application would fail also creating a denial of service attack.

	Potential areas of attack
	\begin{itemize}
		\item Stop the Listener -- create a DoS attack.
		\item Set a password and prevent others from controlling the Listener - Hijack the DB.
		\item Write trace and log files to any file accessible to the process owner of tnslnsr 
		- Possible informationleakage.
		\item Obtain detailed information on the Listener, database, and application configuration.
	\end{itemize}



\section{Infrastructure Configuration Management Testing}
	Proper configuration management of the web server infrastructure is very important in order 
	to preserve the security of the application itself. If elements such as the web server 
	software, the back-end database servers, or the authentication servers are not properly reviewed 
	and secured, they might introduce undesired risks or introduce new vulnerabilities that might
	compromise the application itself.

	In order to test the configuration management infrastructure, the following steps need to be taken:
		\begin{itemize}
			\item The different elements that make up the infrastructure need to be determined in 
			order to understand how they interact with a web application and how they affect 
			its security.
			\item All the elements of the infrastructure need to be reviewed in order to make sure 
			that they don’t hold any known vulnerabilities.
			\item A review needs to be made of the administrative tools used to maintain all the 
			different elements.
			\item The authentication systems, if any, need to reviewed in order to assure that 
			they serve the needs of the application and that they cannot be manipulated by external 
			users to leverage access.
			\item A list of defined ports which are required for the application should be maintained 
			and kept under change control.
		\end{itemize}


\section{Application Configuration Management Testing}
	Proper configuration of the single elements that make up an application architecture is important 
	in order to prevent mistakes that might compromise the security of the whole architecture.

	{\bf Sample/known files and directories}\\
	Many web servers and application servers provide, in a default installation, sample applications 
	and files that are provided for the benefit of the developer and in order to test that the server 
	is working properly right after installation. However, many default web server applications have 
	been later known to be vulnerable.

	{\bf Comment review} \\
	It is very common, and even recommended, for programmers to include detailed comments on their 
	source code in order to allow for other programmers to better understand why a given decision 
	was taken in coding a given function. Programmers usually do it too when developing large 
	web-based applications. However, comments included inline in HTML code might reveal to a 
	potential attacker internal information that should not be available to them. Sometimes, even
	source code is commented out since a functionality is no longer required, but this comment is 
	leaked out to the HTML pages returned to the users unintentionally. Comment review should be 
	done in order to determine if any information is being leaked through comments.

	{\bf Configuration review}
	The web server or application server configuration takes an important role in protecting the 
	contents of the site and it must be carefully reviewed in order to spot common configuration 
	mistakes. Obviously, the recommended configuration varies depending on the site policy, 
	and the functionality that should be provided by the server software. In most cases, however,
	configuration guidelines (either provided by the software vendor or external parties) should 
	be followed in order to determine if the server has been properly secured. It is impossible 
	to generically say how a server should be configured, however, some common guidelines should 
	be taken into account:

		\begin{itemize}
			\item Only enable server modules that are needed for the application. This reduces 
			the attack surface since the server is reduced in size and complexity as software 
			modules are disabled.
			\item Handle server errors with custom-made pages instead of with the default web 
			server pages. Specifically make sure that any application errors will not be returned 
			to the end-user and that no code is leaked through these since it will help an attacker. 
			\item Make sure that the server software runs with minimised privileges in the operating 
			system. 
			\item Make sure the server software logs properly both legitimate access and errors.
			\item Make sure that the server is configured to properly handle overloads and prevent 
			Denial of Service attacks. Ensure that the server has been performance tuned properly.
		\end{itemize}

	{\bf Logging} \\
	Logging is an important asset of the security of an application architecture, since it can be used 
	to detect flaws in applications as well as sustained attacks from rogue users. 
	In both cases (server and application logs) several issues should be tested and analysed based on 
	the log contents:
		\begin{itemize}
			\item Are the logs stored in a dedicated server?
			\item Can log usage generate a Denial of Service condition?
			\item How are they rotated? Are logs kept for the sufficient time?
			\item Do the logs contain sensitive information?
			\item How are logs reviewed? Can administrators use these reviews to detect 
			targeted attacks?
			\item How are log backups preserved?
			\item Is the data being logged data validated (min/max length, chars etc) 
			prior to being logged?
		\end{itemize}

\section{Testing for File Extensions Handling}
	File extensions are commonly used in web servers to easily determine which technologies 
	/ languages / plugins must be used to fulfill the web request.

	Determining how web servers handle requests corresponding to files having different extensions 
	may help to understand web server behaviour depending on the kind of files we try to access. 
	For example, it can help understand which file extensions are returned as text/plain versus 
	those which cause execution on the server side. The latter are indicative of technologies / 
	languages / plugins which are used by web servers or application servers, and may provide 
	additional insight on how the web application is engineered. For example, a “.pl” extension 
	is usually associated with server-side Perl support.

	The following file extensions should NEVER be returned by a web server, since they are related 
	to files which may contain sensitive information, or to files for which there is no reason to 
	be served.
		\begin{itemize}
			\item .asa
			\item .inc
		\end{itemize}

	The following file extensions are related to files which, when accessed, are either displayed 
	or downloaded by the browser. Therefore, files with these extensions must be checked to verify 
	that they are indeed supposed to be served (and are not leftovers), and that they do not contain
	sensitive information.
		\begin{itemize}
			\item {\bf .java:} No reason to provide access to Java source files
			\item {\bf .txt:} Text files
			\item {\bf .zip, .tar, .gz, .tgz, .rar, ...:} (Compressed) archive files
			\item {\bf .pdf:} PDF documents
			\item {\bf .doc, .rtf, .xls, .ppt, ...:} Office documents
			\item {\bf .bak, .old} and other extensions indicative of backup files 
			(for example: ~ for Emacs backup files)
		\end{itemize}

\section{Old, Backup and Unreferenced Files}

	While most of the files within a web server are directly handled by the server itself it 
	isn't uncommon to find {\bf unreferenced and/or forgotten files} that can be used to obtain 
	important information about either the infrastructure or the credentials. Most common 
	scenarios include the presence of renamed old version of modified files, inclusion files 
	that are loaded into the language of choice and can be downloaded as source, or even 
	automatic or manual backups in form of compressed archives.

	An important source of vulnerability lies in {\bf files which have nothing to do with the 
	application}, but are created as a consequence of editing application files, or after 
	creating on-the-fly backup copies, or by leaving in the web tree old files or unreferenced 
	files. Performing in-place editing or other administrative actions on production web servers 
	may inadvertently leave, as a consequence, backup copies.

	As a result, these activities generate files which 
	a) are not needed by the application, 
	b) may be handled differently than the original file by the web server. 

	For example, {\bf if we make a copy of login.asp named login.asp.old}, we are allowing users to
	download the source code of login.asp; this is because, due to its extension, login.asp.old 
	will be typically served as text/plain, rather than being executed. In other words, accessing 
	login.asp causes the execution of the server-side code of login.asp, while accessing 
	login.asp.old causes the content of login.asp.old (which is, again, server-side code) to be 
	plainly returned to the user – and displayed in the browser. This may pose security risks, 
	since sensitive information may be revealed. Generally, exposing server side code is a bad 
	idea; not only are you unnecessarily exposing business logic, but you
	may be unknowingly revealing application-related information which may help an attacker, 
	not to mention the fact that there are too many scripts with embedded username/password 
	in clear text (which is a careless and very dangerous practice).

	Other causes of unreferenced files are due to design or configuration choices when they 
	allow diverse kind of application- related files such as data files, configuration files, 
	log files, to be {\bf stored in filesystem directories} that can be accessed by the
	web server. These files have normally no reason to be in a filesystem space which could 
	be accessed via web, since they should be accessed only at the application level, by the 
	application itself (and not by the casual user browsing around!).

	{\bf Countermeasures}
	To guarantee an effective protection strategy, testing should be compounded by a security 
	policy which clearly forbids dangerous practices, such as:

		\begin{itemize}
			\item Editing files in-place on the web server / application server filesystem. 
			This is a particular bad habit, since it is likely to unwillingly generate backup 
			files by the editors. It is amazing to see how often this is done, even in large
			organizations.
			\item Check carefully any other activity performed on filesystems exposed by the 
			web server, such as spot administration activities. For example, if you occasionally 
			need to take a snapshot of a couple of directories (which you shouldn’t, on a 
			production system...), you may be tempted to zip/tar them first. Be careful not to 
			forget behind those archive files!
			\item Appropriate configuration management policies should help not to leave around 
			obsolete and unreferenced files.
			\item Applications should be designed not to create (or rely on) files stored under 
			the web directory trees served by the web server. Data files, log files, configuration 
			files, etc. should be stored in directories not accessible by the web server, to 
			counter the possibility of information disclosure
		\end{itemize}


\section{Infrastructure and Application Admin Interfaces}

	Administrator interfaces may be present in the application or on the application server to allow 
	certain users to undertake privileged activities on the site. Tests should be undertaken to 
	reveal if and how this privileged functionality can be accessed by an unauthorized or standard 
	user. An application may require an administrator interface to enable a privileged user to access
	functionality that may make changes to how the site functions. Such changes may include:
		\begin{itemize}
			\item User account provisioning
			\item Site design and layout
			\item Data manipultion
			\item Configuration changes
		\end{itemize}

	In many instances, such interfaces are usually implemented with little thought of how to separate 
	them from the normal users of the site. 

\section{Testing for HTTP Methods and XST}
	HTTP offers a number of methods that can be used to perform actions on the web server. Many of 
	theses methods are designed to aid developers in deploying and testing HTTP applications. 
	These HTTP methods can be used for nefarious purposes if the web server is misconfigured. 
	Additionally, Cross Site Tracing (XST), a form of cross site scripting using the servers 
	HTTP TRACE method, is examined.

	the Hypertext Transfer Protocol (HTTP) allows several other methods. 
	Here are the following eight methods: {\bf GET, POST, PUT, DELETE, TRACE, HEAD, 
	OPTIONS and CONNECT}

	Some of these methods can potentially pose a security risk for a web application, as they allow 
	an attacker to modify the files stored on the web server and, in some scenarios, steal the 
	credentials of legitimate users. More specifically, the methods that should be disabled are 
	the following:
		\begin{itemize}
			\item {\bf PUT:} This method allows a client to upload new files on the web server. 
			\item {\bf DELETE:} This method allows a client to delete a file on the web server. 
			\item {\bf CONNECT:} This method could allow a client to use the web server as a proxy
			\item {\bf TRACE:} This method simply echoes back to the client whatever string has been sent 
			to the server, and is used mainly for debugging purposes.
		\end{itemize}
		