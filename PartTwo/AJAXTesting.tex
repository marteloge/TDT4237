\chapter{AJAX Testing}

	{\bf AJAX}, an acronym for {\bf Asynchronous JavaScript and XML}, is a web development technique 
	used to create more responsive web applications. It uses a combination of technologies in order 
	to provide an experience that is more like using a desktop application. This is accomplished by 
	using the XMLHttpRequest object and JavaScript to make asynchronous requests to the web server, 
	parsing the responses and then updating the page DOM HTML and CSS.

	Testing AJAX applications can be challenging because developers are given a tremendous amount of 
	freedom in how they communicate between the client and the server. In traditional web 
	applications, standard HTML forms submitted via GET or POST requests have an easy-to-understand 
	format, and it is therefore easy to modify or create new well-formed requests. AJAX applications 
	often use different encoding or serialization schemes to submit POST data making it difficult for 
	testing tools to reliably create automated test requests.

	Asynchronous Javascript and XML (AJAX) is one of the latest techniques used by web application 
	developers to provide a user experience similar to that of a local application. Since AJAX is 
	still a new technology, there are many security issues that have not yet been fully researched.
	 
	{\bf Some of the security issues in AJAX include:}
		\begin{itemize}
			\item Increased attack surface with many more inputs to secure
			\item Exposed internal functions of the application
			\item Client access to third-party resources with no built-in security and encoding mechanisms
			\item Failure to protect authentication information and sessions
			\item Blurred line between client-side and server-side code, resulting in security mistakes
		\end{itemize}


{\color{red} **NOT FINISHED. IS THE CHAPTER VERY RELEVANT?**}
