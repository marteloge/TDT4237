\chapter{Business Logic Testing}

	Testing for business logic flaws in a multi-functional dynamic web application requires thinking 
	in unconventional ways. This type of vulnerability cannot be detected by a vulnerability scanner 
	and relies upon the skills and creativity of the penetration tester. In addition, this type of
	vulnerability is usually one of the hardest to detect, but, at the same time, usually one of the 
	most detrimental to the application, if exploited. Attacks on the business logic of an application 
	are dangerous, difficult to detect, and are usually specific to the application being tested.

	{\bf Business logic may include:}
	\begin{itemize}
		\item {\bf Business rules} that express business policy (such as channels, location, logistics, 
		prices, and products);
		\item  {\bf Workflows} based on the ordered tasks of passing documents or data from one participant 
		(a person or a software system) to another. 
	\end{itemize}


	Business logic can have security flaws that allow a user to do something that isn't allowed by the
	business. For example, if there is a limit on reimbursement of \$1000, could an attacker misuse the 
	system to request more money than it is intended? Or, perhaps, users are supposed to do operations 
	in a particular order, but an attacker could invoke them out of sequence. Or can a user make a 
	purchase for a negative amount of money?

	{\bf Example 1:}\\
	Setting the quantity of a product on an e-commerce site as a negative number may result in funds 
	being credited to the attacker. The countermeasure to this problem is to implement stronger data
	validation, as the application permits negative numbers to be entered in the quantity field of 
	the shopping cart.

	{\bf Example 2:} \\
	Another more complex example pertains to a commercial financial application that a large institution 
	uses for their business customers. When a user is created by an administrator account, a new userid 
	is associated with this new account. The userids that are created are predictable. For example, 
	if an admin from a fictitious customer of "Spacely Sprockets" creates two accounts consecutively, 
	their respective userids will be 115 and 116. To make things worse, if two more accounts are created 
	and their respective userids are 117 and 119, then it can be assumed that another company's admin 
	has created a user account for their company with the userid of 118.

	{\bf Creating Raw Data for Designing Logical Tests} \\
		\begin{itemize}
			\item {\bf All application business scenarios:} Checkout, Browse, Product ordering, Search for a 
			product, etc.
			\item {\bf Workflows}
			\item {\bf Different user roles:} Manager, Staff, Administrator, CEO, etc.
			\item {\bf Different groups or departments:} Purchasing, Marketing, Engineering, etc.
			\item {\bf Access Rights of Various User Roles and Groups}
			\item {\bf Privilege Table}
		\end{itemize}


