\chapter{Web Services Testing}

	{\bf SOA (Service Orientated Architecture)/Web services applications} are up-and-coming systems 
	which are enabling businesses to interoperate and are growing at an unprecedented rate. 
	Webservice "clients" are generally not user web front-ends but other backend servers. 
	Webservices are exposed to the net like any other service but can be used on HTTP, FTP, SMTP, MQ
	among other transport protocols. The Web Services Framework utilizes the HTTP protocol 
	(as standard Web Application) in conjunction with XML, SOAP, WSDL and UDDI technologies:
		\begin{itemize}
			\item {\bf The "Web Services Description Language" (WSDL)} is used to describe the interfaces 
			of a service.
			\item {The "Simple Object Access Protocol" (SOAP)} provides the means for communication between 
			Web Services and Client Applications with XML and HTTP.
			\item {\bf "Universal Description, Discovery and Integration" (UDDI)} is used to register and
			publish Web Services and their characteristics so that they can be found from potential 
			clients.
		\end{itemize}

	The vulnerabilities in web services are similar to other vulnerabilities, such as SQL injection,
	information disclosure, and leakage, but web services also have unique XML/parser related 
	vulnerabilities, 

	\section{WS Information Gathering}
	{\bf Zero Knowledge} \\
	Normally you will have a WSDL path to access the Web Service, but if you have zero knowledge about it, 
	you will have to use UDDI to find a specific service. 

	{\bf WSDL endpoints} \\
	When a tester accesses the WSDL, he can determine an access point and available interfaces for web
	services. These interfaces or methods take inputs using SOAP over HTTP/HTTPS. If these inputs are 
	not defined well at the source code level, they can be compromised and exploited. 

	\section{HTTP GET Parameters/REST Testing}

		Many XML applications are invoked by passing them parameters using HTTP GET queries. 
		These are sometimes known as “REST-style" Web Services 
		{\bf (REST = Representational State Transfer)}. These Web Services can be attacked by 
		passing malicious content on the HTTP GET string. 

	\section{Replay Testing}

		This section describes testing replay vulnerabilities of a web service. The threat for a replay 
		attack is that the attacker can assume the identity of a valid user and commit some nefarious act
		without detection. A replay attack is a "man-in-the-middle" type of attack where a message is 
		intercepted and replayed by an attacker to impersonate the original sender. For web services, as 
		with other types of HTTP traffic, a sniffer such as Ethereal or Wireshark can capture traffic 
		posted to a web service and using a tool like WebScarab, a tester can resend a packet to the
		target server. An attacker can attempt to resend the original message or change the message in 
		order to compromise the host server.

{\color{red} **NOT FINISHED. IS THE CHAPTER VERY RELEVANT?**}



