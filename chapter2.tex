\chapter{Usability and Psychology}

{\bf Social engineering}

\clearpage
\section{Attacks Based on Psychology}

	\subsection{Pretexting}
	Pretexting is a form of social engineering in which an individual lies to obtain privileged data. 
	A pretext is a false motive. Pretexting often involves a scam where the liar pretends to need 
	information in order to confirm the identity of the person he is talking to. After establishing 
	trust with the targeted individual, the pretexter might ask a series of questions designed to 
	gather key individual identifiers.

	The pretexter's goal is to obtain personal information about you, such as your SSN, your bank or 
	credit card account numbers, mother's maiden name, information contained in your credit report, 
	or the existence and size of your savings and investment portfolios.
	After getting your answers, the pretexter may call your financial institution pretending to 
	be you or someone with authorized access to your account. The pretexter may, for example, 
	claim that he's forgotten his checkbook and needs information about his account.

	\begin{figure}[H]
		\includegraphics[width=\textwidth]{pics/pretexting.png}
	\end{figure}

	\clearpage
	\subsection{Phising}
	Phishing is a technique of fraudulently obtaining private information. Typically, the phisher sends 
	an e-mail that appears to come from a legitimate business — a bank, or credit card company—requesting
	"verification" of information and warning of some dire consequence if it is not provided. The e-mail 
	usually contains a link to a fraudulent web page that seems legitimate — with company logos and content
	and has a form requesting everything from a home address to an ATM card's PIN.

	\begin{figure}[H]
		\includegraphics[width=\textwidth]{pics/phishing.png}
		\caption{Phising attempt}
	\end{figure}

\clearpage
\section{Insights from Psychology Research}
	
	\subsection*{What the Brain does Worse Than the Computer}
	\begin{itemize}
		\item Actions performed often become a matter of skill, but this comes with
		a downside: inattention can cause a practised action to be performed
		instead of an intended one. We are all familiar with such capture errors;
		an example is when you intend to go to the supermarket on the way home from 
		work but take the road home by mistake — as that’s what you do most days. 
		In computer systems, people are trained to click ‘OK’ to pop-up boxes as that’s 
		often the only way to get the work done; some attacks have used the fact that 
		enough people will do this even when they know they shouldn’t.
		\item Actions that people take by following rules are open to errors when 
		they follow the wrong rule. Various circumstances — such as information overload 
		— can cause people to follow the strongest rule they know, or the most general rule, 
		rather than the best one. Examples of phishermen getting people to follow the wrong 
		rule include using https (because ‘it’s secure’) and starting URLs with the 
		impersonated bank’s name, as www.citibank.secureauthentication.com — 
		looking for the name being for many people a stronger rule than parsing its position.
		\item The third category of mistakes are those made by people for cognitive
		reasons — they simply don’t understand the problem. For example, Microsoft’s latest 
		(IE7) anti-phishing toolbar is easily defeated by a picture-in-picture attack. 
	\end{itemize}

	\begin{figure}[H]
		\centering
		\includegraphics[scale=0.8]{pics/insecureweb.png}
		\caption{Actions performed often become a matter of skill}
	\end{figure}

	\clearpage
	\subsection*{Preceptual Bias and Behavioural Economics}
		\begin{itemize}
			\item How do people make decisions faced with uncertainty?
			\item By framing an action as a gain rather than as a loss makes people more 
			likely to take it. 
			\item We’re also bad at calculating probabilities, and use all sorts of heuristics 
			to help us make decisions: we base inferences on familiar or easily-imagined analogies, 
			and by comparison with recent experiences.
			\item We also worry too much about unlikely events. We’re morelikely to be sceptical 
			about things we’ve heard than about things we’ve seen.
			\item Many frauds work by appealing to our atavistic instincts to trust people 
			more in certain situations or over certain types of decision.
			\item Many people perceive terrorism to be a much worse threat than food poisoning 
			or road traffic accidents: this is irrational.
			We overestimate the small risk of dying in a terrorist attack not just 
			because it’s small but because of the visual effect of the 9/11 on TV.
			\item Global warming doesn’t violate anyone’s moral sensibilities and
			it’s a long-term threat rather than a clear and present danger. Humans are 
			sensitive to rapid changes in the environment rather than slow ones.
			\item We are less afraid when we’re in control, such as when driving a car, as opposed 
			to being a passenger in a car or airplane and we are more afraid of 
			uncertainty, that is, when the magnitude of the risk is unknown.
			\item We have a tendency to plump for the alternative that’s ‘good enough’ rather than 
			face the cognitive strain of trying to work out the odds perfectly. Another
			aspect of this is that many people just plump for the standard configuration
			of a system, as they assume it will be good enough. This is one reason why
			secure defaults matter.
			\item The appearance of protection can matter just as much as the reality. For example,
			many people don't use electronic banking because of a fear of fraud, so banks pay
			a fortune for the time of branch and call-center staff. It's not enough for the security engineer to stop bad things happening; you also have to reassure people.
		\end{itemize} 

		\begin{figure}[H]
			\centering
			\includegraphics[scale=0.5]{pics/uncertainty.jpg}
		\end{figure}

	\subsection*{Differences Between People - Gender}
		Most information systems are designed by men, and yet over half their
		users may be women. Recently people have realised that software can create
		barriers to females, and this has led to research work on ‘gender HCI’ — on
		how software should be designed so that women as well as men can use
		it effectively. For example, it’s known that women navigate differently from
		men in the real world, using peripheral vision more, and it duly turns
		out that larger displays reduce gender bias. Other work has focused on
		female programmers, especially end-user programmers working with tools
		like spreadsheets. It turns out that women tinker less than males, but more
		effectively. 

		Women appear to be more thoughtful, but lower self-esteem and
		higher risk-aversion leads them to use fewer features. Given that many of the
		world’s spreadsheet users are women, this work has significant implications
		for product design.

		Simon Baron-Cohen, classifies human brains into type S (systematizers) and 
		type E (empathizers). Type S people are better at geometry and some 
		kinds of symbolic reasoning, while type E's are better at language and 
		multiprocessing. Most men are type S, while most women are type E, a 
		relationship that Baron-Cohen believes is due to
		fetal testosterone levels. 

		\begin{figure}[H]
			\centering
			\includegraphics[scale=0.4]{pics/gender.jpg}
			\caption{Are security mechanisms gender-neutral?}
		\end{figure}

	\clearpage
	\subsection*{Social Psychology}
		This discipline attempts to explain how the thoughts, feelings, and behaviour
		of individuals are influenced by the actual, imagined, or implied presence of
		others. It has many aspects, from the identity that people derive from belonging
		to groups, through the self-esteem we get by comparing ourselves with others.

		The growth of social-networking systems will
		lead to peer pressure being used as a tool for deception, just as it is currently
		used as a tool for marketing fashions.

		Experiments has showed that people will obey an authority rather than their conscience. 
		Most people will do downright immoral things if they are told to.
		A Experiment showed that normal people can behave wickedly even in the absence of orders. 
		In 1971, experimenter Philip Zimbardo set up a ‘prison’ at Stanford where 24 students
		were assigned at random to the roles of 12 warders and 12 inmates. The aim
		of the experiment was to discover whether prison abuses occurred because
		warders (and possibly prisoners) were self-selecting. However, the students
		playing the role of warders rapidly became sadistic authoritarians.
		Abuse of authority, whether real or ostensible, is a major issue for people
		designing operational security measures. Abuse of authority are the key in 
		in security experiments. 

	\begin{figure}[H]
		\centering
		\includegraphics[scale=0.5]{pics/authority.png}
	\end{figure}


	\subsection*{What the Brain Does Better Than the Computer}
		\begin{itemize}
			\item We are extremely good at recognising other humans visually, an ability shared by 
			many primates.
		
			\item We are good at image recognition generally; a task such as ‘pick out all scenes
			in this movie where a girl rides a horse next to water’ is trivial for a human
			child yet a hard research problem in image processing. 

			\item We’re also better thanmachines at understanding speech, particularly in noisy 
			environments, and at identifying speakers.

			\item These abilities mean that it’s possible to devise tests that are easy for humans
			to pass but hard for machines — the so-called ‘CAPTCHA’ test.
		\end{itemize}

\clearpage
\section{Passwords}

	It’s hard to think of a worse authentication mechanism than passwords, given
	what we know about human memory: people can’t remember infrequently-
	used, frequently-changed, or many similar items; we can’t forget on demand;
	recall is harder than recognition; and non-meaningful words are more difficult.

	There are system and policy issues too: as people become principals in more
	and more electronic systems, the same passwords get used over and over
	again. Not only may attacks be carried out by outsiders guessing passwords,
	but by insiders in other systems. People are now asked to choose passwords
	for a large number of websites that they visit rarely.

	Passwords are not, of course, the only way of authenticating users to
	systems. There are basically three options. The person may retain physical
	control of the device — as with a remote car door key. The second is that
	she presents something she knows, such as a password. The third is to use
	something like a fingerprint or iris pattern
	
	Passwords matter, and managing them is a serious real world problem
	that mixes issues of psychology with technical issues. There are basically three
	broad concerns, in ascending order of importance and difficulty:
	
		\begin{enumerate}
			\item Will the user enter the password correctly with a high enough
			probability?
			\item Will the user remember the password, or will they have to either write it
			down or choose one that’s easy for the attacker to guess?
			\item Will the user break the system security by disclosing the password
			to a third party, whether accidentally, on purpose, or as a result of
			deception?
		\end{enumerate}


\clearpage
\subsection{Difficulties with Remembering the Password}

	Our first human-factors issue is that if a password is too long or complex,
	users might have difficulty entering it correctly.

	when customers are expected to memorize passwords, they either choose values which are easy for
	attackers to guess, or write them down, or both. In fact, the password problem
	has been neatly summed up as: ‘‘Choose a password you can’t remember, and
	don’t write it down.’’

	People choose naive passwords. People will use names, single letters, or even just hit carriage return
	giving an empty string as their password. So some systems started to require
	minimum password lengths, or even check user entered passwords against a
	dictionary of bad choices. However, password quality enforcement is harder
	than you might think. 



\subsection{User Abilities, Training and Design Errors}
	So what can you realistically expect from users when it comes to
	choosing and remembering passwords?

	After a experiment (page 35-36) on password an humans, the conclusion was:
	\begin{itemize}
		\item For users who follow instructions, passwords based on mnemonic
		phrases offer the best of both worlds. They are as easy to remember as
		naively selected passwords, and as hard to guess as random passwords.
		\item The problem then becomes one of user compliance. A significant number
		of users (perhaps a third of them) just don’t do what they’re told.
	\end{itemize}

	An important example of how not to do it is to ask for ‘your mother’s
	maiden name’. A surprising number of banks, government departments and
	other organisations authenticate their customers in this way. But there are two
	rather obvious problems. First, your mother’s maiden name is easy for a thief
	to find out, whether by asking around or using online genealogical databases.
	Second, asking for a maiden name makes assumptions which don’t hold for
	all cultures, so you can end up accused of discrimination: Icelanders have no
	surnames, and women from many other countries don’t change their names
	on marriage. Third, there is often no provision for changing ‘your mother’s
	maiden name’, so if it ever becomes known to a thief your customer would
	have to close bank accounts (and presumably reopen them elsewhere). 

	A general error is failing to reset the default password in systems. 
	Often is the default password the same in many different systems!

	Banks have put som effort into trying to train their customers to look 
	for certain features in websites: check the english, look for the lock symbol, 
	it is ok to click on images, but not on URL's, hover your mouse over links 
	before clicking on it, etc. This sort of arms race is more likely to benefit
	the attackers because the customers get very predictable in how they act.

\subsection{Social-Engineering Attacks}

	‘This is the security department of your bank. We see that your card
	has been used fraudulently to buy gold coins. I wonder if you can tell me the
	PIN, so I can get into the computer and cancel it?’. This even works by email!

	A huge problem is that many banks and other businesses train their customers to act in
	unsafe ways. It’s not prudent to click on links in emails, so if you want to
	contact your bank you should type in the URL or use a bookmark — yet bank
	marketing departments continue to send out emails containing clickable links.
	Many email clients — including Apple’s, Microsoft’s, and Google’s — make
	plaintext URLs clickable, and indeed their users may never see a URL that
	isn’t. This makes it harder for banks to do the right thing.

	If even the fraud department doesn’t understand that banks ought to be able to 
	identify themselves, and that customers should not be trained to give out 
	security information on the phone, what hope is there? When even a bank’s 
	security department can’t tell spam from phish, how are their customers supposed to?

	\clearpage
	{\bf Trusted Path:} A thread in the background of phishing is trusted path, 
	which refers to some means of being sure that you’re logging into a genuine machine
	through a channel that isn’t open to eavesdropping. Here the deception is
	more technical than psychological; rather than inveigling a bank customer into
	revealing her PIN to you by claiming to be a policeman, you steal her PIN
	directly by putting a false ATM in a shopping mall. A public
	terminal would be left running an attack program that looks just like the usual
	logon screen — asking for a user name and password. When an unsuspecting
	user does this, it will save the password somewhere in the system, reply
	‘sorry, wrong password’ and then vanish, invoking the genuine password
	program. The user will assume that he made a typing error first time and
	think no more of it. 

	Have you wondered why Windows have the anoying ctrl-alt-del part befor login?
	It is there to make sure you will be entering genuine password prompt.

	Other similar attacks is mofifoed keyboard, keyloggers, and skimmed bank terminals.
	I sometimes wonder, what if there is a keylogger in a windows system?
	If you press ctrl-alt-del, you will simply tell the attacker where your password is.
	The logged txt file will then have something like "ctrl alt del username password".

	Two-factor authentication is now widley used. The basic idea is that it consists
	of something you have and something you know. It will be described in later chapters. 

\clearpage
\section{System Issues}

	There are also technical issues to do with password entry and storage that will 
	be briefly covered here. Here are some examples of attacks we are trying to defend against:

	\begin{itemize}
		\item Targeted attack on one account (when sending email it is known as spear phishing)
		\item Attempt to penetrate any account on a system
		\item Service denial attack
	\end{itemize}

	{\bf Can you deny service?} Banks often have a rule that a terminal and user account
	are frozen after three bad password attempts; after that, an administrator has to 
	reactivate them. It sure works in a website, but what in the military? Could a system
	like this benefit the enemy/attacker? Many commercial websites nowadays don't limit guessing
	becuase of the possibility of such an attack (service denial attack).

	{\bf Interface design: } I usually cover my dialling hand with my body or my other hand when
	entering a card number or PIN in a public place — but you shouldn’t design
	systems on the assumption that all your customers will do this. Many people
	are uncomfortable shielding a PIN from others as it’s a visible signal of distrust.
	Because of this, a targeted attack against a person could be easy!

	{\bf Eavesdropping: } The latest modus operandi is for bad people
	to offer free WiFi access in public places, and harvest the passwords that
	users enter into websites. This is a kind of attempt to just get all sensitive information
	about anyone, and not just a specific person. 

	{\bf Timing attack: }Many kids find out that a bicycle combination lock can usually be 
	broken in a few minutes by solving each ring in order of looseness. The same idea
	worked against a number of computer systems. The PDP-10 TENEX operating
	system checked passwords one character at a time, and stopped as soon as
	one of them was wrong. This opened up a timing attack: the attacker would
	repeatedly place a guessed password in memory at a suitable location, have it
	verified as part of a file access request, and wait to see how long it took to be
	rejected


	{\bf Car keys (fail):} Have you ever heard a story about someone that could open 
	many cars at a parking lot with their own key? Yes, this is a kind of system fails
	where the number of bits used for opening a car was to small in order of how many
	cars that needed an uniqe key.

	{\bf One-way encryption and plaintext storage:} Password storage has also been a problem 
	for some systems. Keeping a plaintext file of passwords can be dangerous. The best way
	of storing a password is password + salt + hashing (can also add pepper to it).

	{\bf Prevent more than x attempts or log all attempts? }

	There are various problems with this doctrine, of which the worst may be that 
	the attacker’s goal is often not to guess some particular user’s password 
	but to get access to any account. If a large defense network has a million 
	possible passwords and a million users, and the alarm goes off after three 
	bad password attempts on any account, then the attack is to try one password 
	for every single account. Thus the quantity of real interest is the probability 
	that the password space can be exhausted in the lifetime of the system at the 
	maximum feasible password guess rate.

	To prevent more than one password guess every few seconds per user account, or 
	(if you can) by source IP address. You might also keep a count of all the failed 
	logon attempts and analyse them: is there a constant series of guesses that could 
	indicate an attempted intrusion? (And what would you do if you noticed one?).

\section{CAPTCHA's}

	Recently people have tried to design protection mechanisms that use the
	brain’s strengths rather than its weaknesses. One early attempt was Passfaces:
	this is an authentication system that presents users with nine faces, only one
	of which is of a person they know; they have to pick the right face several
	times in a row to log on [356]. The rationale is that people are very good
	at recognising other people’s faces, but very bad at describing them: so you
	could build a system where it was all but impossible for people to give away
	their passwords, whether by accident or on purpose. Other proposals of this
	general type have people selecting a series of points on an image — again,
	easy to remember but hard to disclose. Both types of system make shoulder
	surfing harder, as well as deliberate disclosure offline.

	The most successful innovation in this field, however, is the CAPTCHA —
	which stands for ‘Completely Automated Public Turing Test to Tell Computers
	and Humans Apart.’

	The idea is that a program generates some random text,
	and produces a distorted version of it that the user must decipher. Humans
	are good at reading distorted text, while programs are less good. 

	It is inspired by the test famously posed by Alan Turing as to whether a computer
	was intelligent, where you put a computer in one room and a human in
	another, and invite a human to try to tell them apart. 

